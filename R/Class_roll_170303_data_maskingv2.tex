\documentclass[]{article}
\usepackage{lmodern}
\usepackage{amssymb,amsmath}
\usepackage{ifxetex,ifluatex}
\usepackage{fixltx2e} % provides \textsubscript
\ifnum 0\ifxetex 1\fi\ifluatex 1\fi=0 % if pdftex
  \usepackage[T1]{fontenc}
  \usepackage[utf8]{inputenc}
\else % if luatex or xelatex
  \ifxetex
    \usepackage{mathspec}
  \else
    \usepackage{fontspec}
  \fi
  \defaultfontfeatures{Ligatures=TeX,Scale=MatchLowercase}
\fi
% use upquote if available, for straight quotes in verbatim environments
\IfFileExists{upquote.sty}{\usepackage{upquote}}{}
% use microtype if available
\IfFileExists{microtype.sty}{%
\usepackage{microtype}
\UseMicrotypeSet[protrusion]{basicmath} % disable protrusion for tt fonts
}{}
\usepackage[margin=1in]{geometry}
\usepackage{hyperref}
\hypersetup{unicode=true,
            pdftitle={Identity Masking with Class Roll Data},
            pdfauthor={coop711},
            pdfborder={0 0 0},
            breaklinks=true}
\urlstyle{same}  % don't use monospace font for urls
\usepackage{color}
\usepackage{fancyvrb}
\newcommand{\VerbBar}{|}
\newcommand{\VERB}{\Verb[commandchars=\\\{\}]}
\DefineVerbatimEnvironment{Highlighting}{Verbatim}{commandchars=\\\{\}}
% Add ',fontsize=\small' for more characters per line
\usepackage{framed}
\definecolor{shadecolor}{RGB}{248,248,248}
\newenvironment{Shaded}{\begin{snugshade}}{\end{snugshade}}
\newcommand{\KeywordTok}[1]{\textcolor[rgb]{0.13,0.29,0.53}{\textbf{#1}}}
\newcommand{\DataTypeTok}[1]{\textcolor[rgb]{0.13,0.29,0.53}{#1}}
\newcommand{\DecValTok}[1]{\textcolor[rgb]{0.00,0.00,0.81}{#1}}
\newcommand{\BaseNTok}[1]{\textcolor[rgb]{0.00,0.00,0.81}{#1}}
\newcommand{\FloatTok}[1]{\textcolor[rgb]{0.00,0.00,0.81}{#1}}
\newcommand{\ConstantTok}[1]{\textcolor[rgb]{0.00,0.00,0.00}{#1}}
\newcommand{\CharTok}[1]{\textcolor[rgb]{0.31,0.60,0.02}{#1}}
\newcommand{\SpecialCharTok}[1]{\textcolor[rgb]{0.00,0.00,0.00}{#1}}
\newcommand{\StringTok}[1]{\textcolor[rgb]{0.31,0.60,0.02}{#1}}
\newcommand{\VerbatimStringTok}[1]{\textcolor[rgb]{0.31,0.60,0.02}{#1}}
\newcommand{\SpecialStringTok}[1]{\textcolor[rgb]{0.31,0.60,0.02}{#1}}
\newcommand{\ImportTok}[1]{#1}
\newcommand{\CommentTok}[1]{\textcolor[rgb]{0.56,0.35,0.01}{\textit{#1}}}
\newcommand{\DocumentationTok}[1]{\textcolor[rgb]{0.56,0.35,0.01}{\textbf{\textit{#1}}}}
\newcommand{\AnnotationTok}[1]{\textcolor[rgb]{0.56,0.35,0.01}{\textbf{\textit{#1}}}}
\newcommand{\CommentVarTok}[1]{\textcolor[rgb]{0.56,0.35,0.01}{\textbf{\textit{#1}}}}
\newcommand{\OtherTok}[1]{\textcolor[rgb]{0.56,0.35,0.01}{#1}}
\newcommand{\FunctionTok}[1]{\textcolor[rgb]{0.00,0.00,0.00}{#1}}
\newcommand{\VariableTok}[1]{\textcolor[rgb]{0.00,0.00,0.00}{#1}}
\newcommand{\ControlFlowTok}[1]{\textcolor[rgb]{0.13,0.29,0.53}{\textbf{#1}}}
\newcommand{\OperatorTok}[1]{\textcolor[rgb]{0.81,0.36,0.00}{\textbf{#1}}}
\newcommand{\BuiltInTok}[1]{#1}
\newcommand{\ExtensionTok}[1]{#1}
\newcommand{\PreprocessorTok}[1]{\textcolor[rgb]{0.56,0.35,0.01}{\textit{#1}}}
\newcommand{\AttributeTok}[1]{\textcolor[rgb]{0.77,0.63,0.00}{#1}}
\newcommand{\RegionMarkerTok}[1]{#1}
\newcommand{\InformationTok}[1]{\textcolor[rgb]{0.56,0.35,0.01}{\textbf{\textit{#1}}}}
\newcommand{\WarningTok}[1]{\textcolor[rgb]{0.56,0.35,0.01}{\textbf{\textit{#1}}}}
\newcommand{\AlertTok}[1]{\textcolor[rgb]{0.94,0.16,0.16}{#1}}
\newcommand{\ErrorTok}[1]{\textcolor[rgb]{0.64,0.00,0.00}{\textbf{#1}}}
\newcommand{\NormalTok}[1]{#1}
\usepackage{longtable,booktabs}
\usepackage{graphicx,grffile}
\makeatletter
\def\maxwidth{\ifdim\Gin@nat@width>\linewidth\linewidth\else\Gin@nat@width\fi}
\def\maxheight{\ifdim\Gin@nat@height>\textheight\textheight\else\Gin@nat@height\fi}
\makeatother
% Scale images if necessary, so that they will not overflow the page
% margins by default, and it is still possible to overwrite the defaults
% using explicit options in \includegraphics[width, height, ...]{}
\setkeys{Gin}{width=\maxwidth,height=\maxheight,keepaspectratio}
\IfFileExists{parskip.sty}{%
\usepackage{parskip}
}{% else
\setlength{\parindent}{0pt}
\setlength{\parskip}{6pt plus 2pt minus 1pt}
}
\setlength{\emergencystretch}{3em}  % prevent overfull lines
\providecommand{\tightlist}{%
  \setlength{\itemsep}{0pt}\setlength{\parskip}{0pt}}
\setcounter{secnumdepth}{0}
% Redefines (sub)paragraphs to behave more like sections
\ifx\paragraph\undefined\else
\let\oldparagraph\paragraph
\renewcommand{\paragraph}[1]{\oldparagraph{#1}\mbox{}}
\fi
\ifx\subparagraph\undefined\else
\let\oldsubparagraph\subparagraph
\renewcommand{\subparagraph}[1]{\oldsubparagraph{#1}\mbox{}}
\fi

%%% Use protect on footnotes to avoid problems with footnotes in titles
\let\rmarkdownfootnote\footnote%
\def\footnote{\protect\rmarkdownfootnote}

%%% Change title format to be more compact
\usepackage{titling}

% Create subtitle command for use in maketitle
\newcommand{\subtitle}[1]{
  \posttitle{
    \begin{center}\large#1\end{center}
    }
}

\setlength{\droptitle}{-2em}

  \title{Identity Masking with Class Roll Data}
    \pretitle{\vspace{\droptitle}\centering\huge}
  \posttitle{\par}
    \author{coop711}
    \preauthor{\centering\large\emph}
  \postauthor{\par}
      \predate{\centering\large\emph}
  \postdate{\par}
    \date{2019-03-09}


\begin{document}
\maketitle

\subsubsection{Data}\label{data}

\begin{Shaded}
\begin{Highlighting}[]
\NormalTok{class_roll <-}\StringTok{ }\KeywordTok{read.xlsx}\NormalTok{(}\StringTok{"../data/class_roll0303.xlsx"}\NormalTok{, }
                        \DataTypeTok{sheetIndex =} \DecValTok{1}\NormalTok{, }
                        \DataTypeTok{startRow =} \DecValTok{2}\NormalTok{, }
                        \DataTypeTok{endRow =} \DecValTok{162}\NormalTok{, }
                        \DataTypeTok{colIndex =} \KeywordTok{c}\NormalTok{(}\DecValTok{3}\OperatorTok{:}\DecValTok{7}\NormalTok{, }\DecValTok{9}\NormalTok{), }
                        \DataTypeTok{colClasses =} \KeywordTok{rep}\NormalTok{(}\StringTok{"character"}\NormalTok{, }\DecValTok{6}\NormalTok{), }
                        \DataTypeTok{encoding =} \StringTok{"UTF-8"}\NormalTok{, }
                        \DataTypeTok{stringsAsFactors =} \OtherTok{FALSE}\NormalTok{)}
\KeywordTok{names}\NormalTok{(class_roll) <-}\StringTok{ }\KeywordTok{c}\NormalTok{(}\StringTok{"dept"}\NormalTok{, }\StringTok{"id"}\NormalTok{, }\StringTok{"name"}\NormalTok{, }\StringTok{"year"}\NormalTok{, }\StringTok{"email"}\NormalTok{, }\StringTok{"cell_no"}\NormalTok{)}
\end{Highlighting}
\end{Shaded}

\subsubsection{학번 가리기}\label{-}

학번은 입학연도를 나타내는 첫 네자리와 개인 식별번호로 구성되어 있다.
여기서,\\
개인식별번호를 ``9999''로 가려보자. \texttt{substr()\ \textless{}-} 을
이용하면 학번의 개인정보를 가리는 일은 한 줄의 코드로 가능하다.

\begin{Shaded}
\begin{Highlighting}[]
\KeywordTok{substr}\NormalTok{(class_roll}\OperatorTok{$}\NormalTok{id, }\DataTypeTok{start =} \DecValTok{5}\NormalTok{, }\DataTypeTok{stop =} \DecValTok{8}\NormalTok{) <-}\StringTok{ "9999"}
\KeywordTok{kable}\NormalTok{(}\KeywordTok{head}\NormalTok{(class_roll))}
\end{Highlighting}
\end{Shaded}

\begin{longtable}[]{@{}llllll@{}}
\toprule
dept & id & name & year & email & cell\_no\tabularnewline
\midrule
\endhead
중국학과 & 20119999 & 강경민 & 4 &
\href{mailto:ssilmido@naver.com}{\nolinkurl{ssilmido@naver.com}} &
010-9164-5954\tabularnewline
전자공학과 & 20119999 & 강경윤 & 4 &
\href{mailto:33169kang@hanmail.net}{\nolinkurl{33169kang@hanmail.net}} &
010-8574-8159\tabularnewline
컴퓨터공학과 & 20179999 & 강보경 & 1 &
\href{mailto:kbk9818@naver.com}{\nolinkurl{kbk9818@naver.com}} &
010-6435-5735\tabularnewline
화학과 & 20149999 & 강소연 & 4 &
\href{mailto:crown_girl@hanmail.net}{\nolinkurl{crown\_girl@hanmail.net}}
& 010-2066-8619\tabularnewline
경영학과 & 20169999 & 강예은 & 2 &
\href{mailto:yeeun423@naver.com}{\nolinkurl{yeeun423@naver.com}} &
010-8820-6892\tabularnewline
경제학과 & 20129999 & 강정우 & 3 &
\href{mailto:jeongugang@gmail.com}{\nolinkurl{jeongugang@gmail.com}} &
010-7499-8710\tabularnewline
\bottomrule
\end{longtable}

\subsubsection{이름 가리기}\label{-}

\texttt{substring()\ \textless{}-}을 이용하면 각 이름의 2번째 글자
이후를 모두 ``ㅇㅇ''으로 대체할 수 있다.

\begin{Shaded}
\begin{Highlighting}[]
\KeywordTok{substring}\NormalTok{(class_roll}\OperatorTok{$}\NormalTok{name, }\DecValTok{2}\NormalTok{) <-}\StringTok{ "ㅇㅇ"}
\KeywordTok{kable}\NormalTok{(}\KeywordTok{head}\NormalTok{(class_roll))}
\end{Highlighting}
\end{Shaded}

\begin{longtable}[]{@{}llllll@{}}
\toprule
dept & id & name & year & email & cell\_no\tabularnewline
\midrule
\endhead
중국학과 & 20119999 & 강ㅇㅇ & 4 &
\href{mailto:ssilmido@naver.com}{\nolinkurl{ssilmido@naver.com}} &
010-9164-5954\tabularnewline
전자공학과 & 20119999 & 강ㅇㅇ & 4 &
\href{mailto:33169kang@hanmail.net}{\nolinkurl{33169kang@hanmail.net}} &
010-8574-8159\tabularnewline
컴퓨터공학과 & 20179999 & 강ㅇㅇ & 1 &
\href{mailto:kbk9818@naver.com}{\nolinkurl{kbk9818@naver.com}} &
010-6435-5735\tabularnewline
화학과 & 20149999 & 강ㅇㅇ & 4 &
\href{mailto:crown_girl@hanmail.net}{\nolinkurl{crown\_girl@hanmail.net}}
& 010-2066-8619\tabularnewline
경영학과 & 20169999 & 강ㅇㅇ & 2 &
\href{mailto:yeeun423@naver.com}{\nolinkurl{yeeun423@naver.com}} &
010-8820-6892\tabularnewline
경제학과 & 20129999 & 강ㅇㅇ & 3 &
\href{mailto:jeongugang@gmail.com}{\nolinkurl{jeongugang@gmail.com}} &
010-7499-8710\tabularnewline
\bottomrule
\end{longtable}

\subsubsection{전화번호 가리기}\label{-}

모바일 폰 번호의 끝 네 자리를 ``xxxx'' 로 대체한다. 정상적으로 번호가
나올 경우 열번째 글자부터 열세번째글자에 해당한다.

\begin{Shaded}
\begin{Highlighting}[]
\KeywordTok{substring}\NormalTok{(class_roll}\OperatorTok{$}\NormalTok{cell_no, }\DecValTok{10}\NormalTok{, }\DecValTok{13}\NormalTok{) <-}\StringTok{ "xxxx"}
\KeywordTok{kable}\NormalTok{(}\KeywordTok{head}\NormalTok{(class_roll))}
\end{Highlighting}
\end{Shaded}

\begin{longtable}[]{@{}llllll@{}}
\toprule
dept & id & name & year & email & cell\_no\tabularnewline
\midrule
\endhead
중국학과 & 20119999 & 강ㅇㅇ & 4 &
\href{mailto:ssilmido@naver.com}{\nolinkurl{ssilmido@naver.com}} &
010-9164-xxxx\tabularnewline
전자공학과 & 20119999 & 강ㅇㅇ & 4 &
\href{mailto:33169kang@hanmail.net}{\nolinkurl{33169kang@hanmail.net}} &
010-8574-xxxx\tabularnewline
컴퓨터공학과 & 20179999 & 강ㅇㅇ & 1 &
\href{mailto:kbk9818@naver.com}{\nolinkurl{kbk9818@naver.com}} &
010-6435-xxxx\tabularnewline
화학과 & 20149999 & 강ㅇㅇ & 4 &
\href{mailto:crown_girl@hanmail.net}{\nolinkurl{crown\_girl@hanmail.net}}
& 010-2066-xxxx\tabularnewline
경영학과 & 20169999 & 강ㅇㅇ & 2 &
\href{mailto:yeeun423@naver.com}{\nolinkurl{yeeun423@naver.com}} &
010-8820-xxxx\tabularnewline
경제학과 & 20129999 & 강ㅇㅇ & 3 &
\href{mailto:jeongugang@gmail.com}{\nolinkurl{jeongugang@gmail.com}} &
010-7499-xxxx\tabularnewline
\bottomrule
\end{longtable}

\subsubsection{전공 단위 이름 가리기}\label{---}

전공 단위 이름은 ``학과'', ``과'', ``학'', ``전공'' 등 매우 다양한
명칭이 있으므로 \texttt{gsub()} 함수의 정규표현(regulasr expression)을
활용하여 ``ㅇㅇ학과'' 와 같은 방식으로 이름을 가릴 수 있다.

\begin{Shaded}
\begin{Highlighting}[]
\NormalTok{class_roll}\OperatorTok{$}\NormalTok{dept <-}\StringTok{ }\KeywordTok{sub}\NormalTok{(}\StringTok{"^.+$"}\NormalTok{, }\StringTok{"ㅇㅇ학과"}\NormalTok{, class_roll}\OperatorTok{$}\NormalTok{dept)}
\KeywordTok{kable}\NormalTok{(}\KeywordTok{head}\NormalTok{(class_roll))}
\end{Highlighting}
\end{Shaded}

\begin{longtable}[]{@{}llllll@{}}
\toprule
dept & id & name & year & email & cell\_no\tabularnewline
\midrule
\endhead
ㅇㅇ학과 & 20119999 & 강ㅇㅇ & 4 &
\href{mailto:ssilmido@naver.com}{\nolinkurl{ssilmido@naver.com}} &
010-9164-xxxx\tabularnewline
ㅇㅇ학과 & 20119999 & 강ㅇㅇ & 4 &
\href{mailto:33169kang@hanmail.net}{\nolinkurl{33169kang@hanmail.net}} &
010-8574-xxxx\tabularnewline
ㅇㅇ학과 & 20179999 & 강ㅇㅇ & 1 &
\href{mailto:kbk9818@naver.com}{\nolinkurl{kbk9818@naver.com}} &
010-6435-xxxx\tabularnewline
ㅇㅇ학과 & 20149999 & 강ㅇㅇ & 4 &
\href{mailto:crown_girl@hanmail.net}{\nolinkurl{crown\_girl@hanmail.net}}
& 010-2066-xxxx\tabularnewline
ㅇㅇ학과 & 20169999 & 강ㅇㅇ & 2 &
\href{mailto:yeeun423@naver.com}{\nolinkurl{yeeun423@naver.com}} &
010-8820-xxxx\tabularnewline
ㅇㅇ학과 & 20129999 & 강ㅇㅇ & 3 &
\href{mailto:jeongugang@gmail.com}{\nolinkurl{jeongugang@gmail.com}} &
010-7499-xxxx\tabularnewline
\bottomrule
\end{longtable}

\subsubsection{e-mail 가리기}\label{e-mail-}

email 주소는 \texttt{@}를 사이에 두고 나뉘어진다. \texttt{gsub()} 함수와
정규표현(regular expression)을 활용하면 email 주소에서 서비스업체명은
그대로 두고 개인 식별이 가능한 이름 부분을 \texttt{user\_name}으로
대체할 수 있다. 160명 중 20명만 랜덤하게 표본추출한다.

\begin{Shaded}
\begin{Highlighting}[]
\NormalTok{class_roll}\OperatorTok{$}\NormalTok{email <-}\StringTok{ }\KeywordTok{sub}\NormalTok{(}\StringTok{"^.+@"}\NormalTok{, }\StringTok{"user_name@"}\NormalTok{, class_roll}\OperatorTok{$}\NormalTok{email)}
\KeywordTok{kable}\NormalTok{(class_roll[}\KeywordTok{sample}\NormalTok{(}\DecValTok{1}\OperatorTok{:}\KeywordTok{nrow}\NormalTok{(class_roll), }\DataTypeTok{size =} \DecValTok{25}\NormalTok{), ])}
\end{Highlighting}
\end{Shaded}

\begin{longtable}[]{@{}lllllll@{}}
\toprule
& dept & id & name & year & email & cell\_no\tabularnewline
\midrule
\endhead
35 & ㅇㅇ학과 & 20169999 & 김ㅇㅇ & 2 &
\href{mailto:user_name@naver.com}{\nolinkurl{user\_name@naver.com}} &
010-5003-xxxx\tabularnewline
68 & ㅇㅇ학과 & 20179999 & 배ㅇㅇ & 1 &
\href{mailto:user_name@naver.com}{\nolinkurl{user\_name@naver.com}} &
010-8259-xxxx\tabularnewline
111 & ㅇㅇ학과 & 20149999 & 이ㅇㅇ & 2 &
\href{mailto:user_name@naver.com}{\nolinkurl{user\_name@naver.com}} &
010-5250-xxxx\tabularnewline
61 & ㅇㅇ학과 & 20149999 & 박ㅇㅇ & 4 &
\href{mailto:user_name@naver.com}{\nolinkurl{user\_name@naver.com}} &
010-9395-xxxx\tabularnewline
1 & ㅇㅇ학과 & 20119999 & 강ㅇㅇ & 4 &
\href{mailto:user_name@naver.com}{\nolinkurl{user\_name@naver.com}} &
010-9164-xxxx\tabularnewline
158 & ㅇㅇ학과 & 20159999 & 황ㅇㅇ & 3 &
\href{mailto:user_name@naver.com}{\nolinkurl{user\_name@naver.com}} &
010-5656-xxxx\tabularnewline
153 & ㅇㅇ학과 & 20149999 & 최ㅇㅇ & 2 &
\href{mailto:user_name@naver.com}{\nolinkurl{user\_name@naver.com}} &
010-9964-xxxx\tabularnewline
27 & ㅇㅇ학과 & 20129999 & 김ㅇㅇ & 4 &
\href{mailto:user_name@hanmail.net}{\nolinkurl{user\_name@hanmail.net}}
& 010-8505-xxxx\tabularnewline
16 & ㅇㅇ학과 & 20139999 & 김ㅇㅇ & 2 &
\href{mailto:user_name@naver.com}{\nolinkurl{user\_name@naver.com}} &
010-2795-xxxx\tabularnewline
12 & ㅇㅇ학과 & 20149999 & 권ㅇㅇ & 2 &
\href{mailto:user_name@naver.com}{\nolinkurl{user\_name@naver.com}} &
010-7208-xxxx\tabularnewline
74 & ㅇㅇ학과 & 20179999 & 손ㅇㅇ & 1 &
\href{mailto:user_name@hanmail.net}{\nolinkurl{user\_name@hanmail.net}}
& 010-9450-xxxx\tabularnewline
147 & ㅇㅇ학과 & 20139999 & 최ㅇㅇ & 3 &
\href{mailto:user_name@naver.com}{\nolinkurl{user\_name@naver.com}} &
010-9079-xxxx\tabularnewline
43 & ㅇㅇ학과 & 20159999 & 김ㅇㅇ & 3 &
\href{mailto:user_name@naver.com}{\nolinkurl{user\_name@naver.com}} &
010-4886-xxxx\tabularnewline
64 & ㅇㅇ학과 & 20159999 & 박ㅇㅇ & 3 &
\href{mailto:user_name@naver.com}{\nolinkurl{user\_name@naver.com}} &
010-5023-xxxx\tabularnewline
77 & ㅇㅇ학과 & 20159999 & 송ㅇㅇ & 3 &
\href{mailto:user_name@naver.com}{\nolinkurl{user\_name@naver.com}} &
010-7711-xxxx\tabularnewline
115 & ㅇㅇ학과 & 20179999 & 이ㅇㅇ & 1 &
\href{mailto:user_name@naver.com}{\nolinkurl{user\_name@naver.com}} &
010-4843-xxxx\tabularnewline
89 & ㅇㅇ학과 & 20149999 & 우ㅇㅇ & 4 &
\href{mailto:user_name@naver.com}{\nolinkurl{user\_name@naver.com}} &
010-5193-xxxx\tabularnewline
71 & ㅇㅇ학과 & 20149999 & 서ㅇㅇ & 2 &
\href{mailto:user_name@naver.com}{\nolinkurl{user\_name@naver.com}} &
010-4099-xxxx\tabularnewline
108 & ㅇㅇ학과 & 20179999 & 이ㅇㅇ & 1 &
\href{mailto:user_name@gmail.com}{\nolinkurl{user\_name@gmail.com}} &
010-9632-xxxx\tabularnewline
100 & ㅇㅇ학과 & 20179999 & 이ㅇㅇ & 1 &
\href{mailto:user_name@naver.com}{\nolinkurl{user\_name@naver.com}} &
010-6351-xxxx\tabularnewline
5 & ㅇㅇ학과 & 20169999 & 강ㅇㅇ & 2 &
\href{mailto:user_name@naver.com}{\nolinkurl{user\_name@naver.com}} &
010-8820-xxxx\tabularnewline
91 & ㅇㅇ학과 & 20179999 & 유ㅇㅇ & 1 &
\href{mailto:user_name@naver.com}{\nolinkurl{user\_name@naver.com}} &
010-2186-xxxx\tabularnewline
14 & ㅇㅇ학과 & 20119999 & 김ㅇㅇ & 3 &
\href{mailto:user_name@naver.com}{\nolinkurl{user\_name@naver.com}} &
010-8009-xxxx\tabularnewline
159 & ㅇㅇ학과 & 20169999 & 황ㅇㅇ & 2 &
\href{mailto:user_name@naver.com}{\nolinkurl{user\_name@naver.com}} &
010-9229-xxxx\tabularnewline
142 & ㅇㅇ학과 & 20139999 & 조ㅇㅇ & 3 &
\href{mailto:user_name@naver.com}{\nolinkurl{user\_name@naver.com}} &
010-4879-xxxx\tabularnewline
\bottomrule
\end{longtable}


\end{document}
